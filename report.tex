\documentclass[journal]{journal}
\usepackage[margin=1in]{geometry}
\usepackage{verbatim}
\usepackage[justification=centering]{subfig}
\usepackage{graphicx}
\usepackage{url} % typeset URL's reasonably
\usepackage{listings}

\usepackage{pslatex} % Use Postscript fonts
\usepackage{algorithm}
\usepackage{algpseudocode}
%\usepackage[]{algorithm2e}
%\usepackage{subcaption}
\usepackage{color}
\usepackage{multirow}
\usepackage{makecell}

\usepackage{mathptmx}
\usepackage{amsmath}

%\usepackage{appendix}
%define some own functions
\newcommand{\tabincell}[2]{\begin{tabular}{@{}#1@{}}#2\end{tabular}} 
\def\D{\mathrm{d}}

\begin{document}

	\title{Towards the Robust Image Recognition Using Spiking Neurons}
	\author{
	Qian~Liu
	\thanks{
	The author is with the School of Computer Science, University of Manchester, Manchester M13 9PL, U.K. 
	(e-mail:qian.liu-3@manchester.ac.uk}
	}
	\maketitle
	\thispagestyle{empty}

\begin{abstract}

\end{abstract}

\section{Introduction}

\begin{appendices}
	\section{Thesis Outline}
	\label{app:thesis}
	The following section-level outline gives the planned thesis structure for this project.
	Sections which are reliant on upcoming work are indicated with a star (*);
	\begin{enumerate}
		\item Introduction
		\item Background
			\begin{enumerate}
				\item Neural Network on Image Recognition
				\item Neuron Models and Spiking Neural Network
				\item Spiking Neural Network Simulation
				\item Neuromorphic Simulators
			\end{enumerate}	
		\item Related Works
			\begin{enumerate}
				\item Vision Databases and Benchmarks
				\item Deep Neural Networks
				\item Spike-Based Image Recognition
				\item Real-Time Neuromorphic Vision System
			\end{enumerate}
		\item Benchmarking Spike-Based Visual Recognition
			\begin{enumerate}
				\item Database
				\item Evaluation Methodology
			\end{enumerate}
		\item Spiking Deep Belief Network
			\begin{enumerate}
				\item Restricted Boltzmann Machine  
				\item Deep Belief Network
				\item Spiking RBM and DBN *
%					\begin{enumerate}
%					\item Mean Field Theory
%					\item Synaptic Learning
%					\end{enumerate}	
			\end{enumerate}	
		\item Benchmarks
			\begin{enumerate}
				\item ConvNet without Learning
				\item STDP Learned 2-Layer Network
				\item Spiking DBN *
			\end{enumerate}	
		\item Discussions *
			\begin{enumerate}
				\item Benefits of Spikes
				\item Scalability of H/W SDBN
				\item Formalisation of SDBN
			\end{enumerate}					
		\item Future Work *
			\begin{enumerate}
				\item SDBN Toolbox on SpiNNaker
				\item Learning on Spiking ConvNet
				\item Video-Based Recognition and Benchmarks
%				\item Benchmarking Real-Time Neuromorphic Systems
			\end{enumerate}					
	\end{enumerate}	
\end{appendices}

\bibliography{ref}    % this causes the references to be listed
\bibliographystyle{ieeetr}
\end{document}